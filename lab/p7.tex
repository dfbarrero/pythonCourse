%%%%%%%%%%%%%%%%%%%%%%%%%%%%%%%%%%%%%%%%%%%%%%%%%%%%%%%%%%%%%%%%%%%%%%%%%%%%%%%
%     enunciado.tex:    Enunciado de la pr�ctica 2 de GIEAI-Inform�tica
%%%%%%%%%%%%%%%%%%%%%%%%%%%%%%%%%%%%%%%%%%%%%%%%%%%%%%%%%%%%%%%%%%%%%%%%%%%%%%%

\documentclass[english,a4paper,11pt]{article}

\usepackage[latin1]{inputenc}  % codificaci�n de caracteres de este archivo
\usepackage[spanish]{babel}    % Traducir: ``abstract'' ---> ``resumen''   etc.
\usepackage{fancyhdr}          % p�ginas con cabecera y pie
\usepackage{listings}          % listados de c�digo fuente
\usepackage{float}             % para que los listados floten como las figuras
\usepackage{vmargin}           % ajuste de m�rgenes f�cil de usar
\usepackage[T1]{fontenc}       % meter fuentes vectoriales
\usepackage{graphicx}          % figuras
\usepackage{upquote}           % comilla recta con \textquotesingle
\usepackage{placeins}          % orden \FloatBarrier para mantener figuras a raya

\usepackage[implicit=false]{hyperref}  % enlaces web (el par�metro implicit=false
                                       % evita problemas con el # de #include etc.)
                                       % uso expl�cito: \href[options]{URL}{text}
                                       %                \url{URL}

% mis abreviaturas
\newcommand{\C}{\texttt{C}}            % escribir� \C en lugar de \texttt{C}
\newcommand{\Pascal}{\texttt{Pascal}}  % idem con \Pascal ...
\newcommand{\fun}[1]{\texttt{#1()}}    % "\fun{main}" ---> "main()" en letra typewriter
\newcommand{\hex}[1]{\texttt{#1}$_{hex}$}
\newcommand{\bin}[1]{\texttt{#1}$_{bin}$}
\newcommand{\enesimo}{\mbox{n-�simo}}
\newcommand{\muvision}{\textit{$\mu\!$Vision4}}
\newcommand{\keilmuvision}{\textit{Keil}~\muvision}
\newcommand{\codigo}[1]{\texttt{#1}}
\newcommand{\menu}[1]{\textit{#1}}

% �rdenes para alternar entre el estilo espa�ol (ligera separaci�n extra
% entre p�rrafos) y el estilo por defecto (p�rrafos junticos junticos)
\newcommand{\parrafosjuntos}{\setlength{\parskip}{0pt}}
\newcommand{\parrafosseparados}{\setlength{\parskip}{1.5ex plus 0.6ex minus 0.5ex}}

% datos importantes del documento
\newcommand{\titulo}{Object-Oriented Programming in Python}     % <<--- T�TULO
\newcommand{\fecha}{Week 7}                         % <<--- TEMA
%\newcommand{\asignatura}{Inteligencia Artificial en los Sistemas de Control Aut�nomo}
\newcommand{\asignatura}{Assignment 7}
\newcommand{\institucion}{UAH, Departamento de Autom�tica, ATC-SOL}
\newcommand{\paginaweb}{http://atc1.aut.uah.es}

% portada
\title{\asignatura \\ \titulo}
\author{\institucion \\ \url{\paginaweb}}
\date{\fecha}

% m�rgenes un poco m�s finos
\setmargrb{25mm}{20mm}{25mm}{20mm}    % left, top, right, bottom

% encabezado y pie
\pagestyle{fancy}
\lhead{\footnotesize \parbox{11cm}{\asignatura}}
\lfoot{\footnotesize \parbox{11cm}{\institucion}}
\cfoot{}
\rhead{\footnotesize \titulo}
\rfoot{\footnotesize P�gina \thepage}
\renewcommand{\headheight}{24pt}
\renewcommand{\footrulewidth}{0.4pt}
%\renewcommand{\headrulewidth}{0pt}

% listados de c�digo fuente flotantes
%\newfloat{floatlisting}{h}{}
%\floatname{floatlisting}{Listado}

% estilo de los listados de c�digo
\lstset{language=Python,
		numbers=left,                 % n�meros de l�nea
        numberstyle=\tiny,            % tama�o de los num. de l�nea
        extendedchars=true,           % acentos, e�es...
        %frame=single,                 % marco que encuadra al listado
        basicstyle=\footnotesize\ttfamily,   % tipo de letra
        showstringspaces=false}       % no mostrar espacios de las cadenas

\graphicspath{{figs/}}                % ruta de las figuras

\begin{document} % ------------------ Aqu� empieza el documento ----------------------

% redefinir el nombre de algunas cosas
\renewcommand{\tablename}{Tabla}                  % mejor "Tabla" que "Cuadro"
\renewcommand{\listtablename}{Indice de tablas}
% definidos originalmene en:
%/usr/share/texmf-texlive/tex/generic/babel/spanish.ldf

\maketitle              % montar el t�tulo aqu�� con los par�metros definidos arriba
\thispagestyle{empty}   % no poner n�mero de p�gina ni nada de nada en la 1� p�gina

\renewcommand{\abstractname}{}         % eliminar "Resumen"
\begin{abstract}                       % resumen (sin la palabra "Resumen")
\noindent \textbf{Objectives:}
\begin{itemize}
\item Create single classes in Python
\item Consume classes
\item Create simple class hierarchies
\item Understand overriding
\item Handle trees with a module
\end{itemize}
\end{abstract}

\sloppy              % hacer m�s flexible el c�lculo del espacio entre palabras para
                     % evitar errores de tipo "overfull box"
                     % (lo contrario de \sloppy es \fussy)

\parrafosseparados   % separaci�n entre p�rrafos (por defecto saldr�n pegados)

Most of the following exercises have been collected from \url{http://introtopython.org/classes.html#what}

%\subsection*{Comentarios iniciales}
\section*{Classes}
\subsection*{Exercise 1}
Modeling a person is a classic exercise for people who are trying to learn how to write classes. We are all familiar with characteristics and behaviors of people, so it is a good exercise to try.
\begin{enumerate}
\item Define a \texttt{Person()} class.
\item In the \texttt{\_\_init()\_\_} function, define several attributes of a person. Good attributes to consider are name, age, place of birth, and anything else you like to know about the people in your life.
\item Write one method. This could be as simple as \texttt{introduce\_yourself()}. This method would print out a statement such as, ``Hello, my name is Eric.''
\item You could also make a method such as \texttt{age\_person()}. A simple version of this method would just add 1 to the person's age.
	\begin{itemize}
	\item A more complicated version of this method would involve storing the person's birthdate rather than their age, and then calculating the age whenever the age is requested. But dealing with dates and times is not particularly easy if you've never done it in any other programming language before.
	\end{itemize}
\item Create a person, set the attribute values appropriately, and print out information about the person.
\item Call your method on the person you created. Make sure your method executed properly; if the method does not print anything out directly, print something before and after calling the method to make sure it did what it was supposed to.
\end{enumerate}

\subsection*{Exercise 2}
Modeling a car is another classic exercise.
\begin{itemize}
\item Define a Car() class.
\item In the \texttt{\_\_init\_\_()} function, define several attributes of a car. Some good attributes to consider are make (Subaru, Audi, Volvo...), model (Outback, allroad, C30), year, num\_doors, owner, or any other aspect of a car you care to include in your class.
\item Write one method. This could be something such as \texttt{describe\_car()}. This method could print a series of statements that describe the car, using the information that is stored in the attributes. You could also write a method that adjusts the mileage of the car or tracks its position.
\item Create a car object, and use your method.
\item Create several car objects with different values for the attributes. Use your method on several of your cars.
\end{itemize}

\section*{Inheritance}

\subsection*{Exercise 3}
Start with your program from \texttt{Person} Class.
\begin{itemize}
\item Make a new class called \texttt{Student} that inherits from \texttt{Person}.
\item Define some attributes that a student has, which other people don't have.
\begin{itemize}
\item A student has a school they are associated with, a graduation year and other particular attributes.
\end{itemize}
\item Create a Student object, and prove that you have used inheritance correctly.
\begin{itemize}
\item Set some attribute values for the student, that are only coded in the Person class.
\item Set some attribute values for the student, that are only coded in the Student class.
\item Print the values for all of these attributes.
\end{itemize}
\end{itemize}
\subsection*{Exercise 4}
Given the following \texttt{Rocket} class:

\lstinputlisting{codep7/Rocket.py}

Perform the following tasks:
\begin{enumerate}
\item Write a \texttt{Shuttle} class, which is just a rocket that can be reused several times. To this end, extend the \texttt{Rocket} class and add the following constructor \texttt{\_\_init()\_\_(self,x=0, y=0, flights\_completed=0)}
\item Create several shuttles and rockets, with random positions random number of flights completed (only for shuttles).
\item Add more attributes that are particular to shuttles such as maximum number of flights, capability of supporting spacewalks, and capability of docking with the ISS.
\item Add one more method to the class, that relates to shuttle behavior. This method could simply print a statement, such as ``Docking with the ISS,'' for a \texttt{dock\_ISS()} method.
\item Prove that your refinements work by creating a \texttt{Shuttle} object with these attributes, and then call your new method.
\end{enumerate}

\section*{Class and modules}

\subsection*{Exercise 5}
Take your program from \texttt{Student} Class
\begin{enumerate}
\item Save your \texttt{Person} and \texttt{Student} classes in a separate file called \texttt{person.py}.
\item Save the code that uses these classes in four separate files.
\begin{enumerate}
\item In the first file, use the \texttt{from module\_name import ClassName} syntax to make your program run.
\item In the second file, use the \texttt{import module\_name} syntax.
\item In the third file, use the \texttt{import module\_name as different\_local\_module\_name} syntax.
\item In the fourth file, use the \texttt{import *} syntax.
\end{enumerate}
\end{enumerate}

\subsection*{Exercise 6}
Take your program from \texttt{Car} Class
\begin{enumerate}
\item Save your \texttt{Car} class in a separate file called \texttt{car.py}.
\item Save the code that uses the car class into four separate files.
\begin{enumerate}
\item In the first file, use the \texttt{from module\_name import ClassName} syntax to make your program run.
\item In the second file, use the \texttt{import module\_name} syntax.
\item In the third file, use the \texttt{import module\_name as different\_local\_module\_name} syntax.
\item In the fourth file, use the \texttt{import *} syntax.
\end{enumerate}
\end{enumerate}

\section*{Trees}
Download the files \texttt{node.py}, \texttt{tree.py} and \texttt{app.py} from \url{http://www.quesucede.com/page/show/id/python-3-tree-implementation}. This is an implementation of the tree data structure.

\begin{enumerate}
\item Read and understand the code in the three files, begining with \texttt{app.py}, which shows an example of how to use a tree.
\item Create a program that reprents the expression $(4+2^{2})*4+7$.
\item Search if there is a node with value $7$ in the previous expression. Repeat the search with the value $34$.
\item Model your family tree from your granfathers, search if ``Jose'' is a member of your family.
\item Which search algorithms are implemented? Locate the code that performs the search.
\end{enumerate}

\end{document}
